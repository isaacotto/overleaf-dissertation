\thesistitle{Scoring the Unknown: Rethinking Fixity and Openness in Western Art Music Notation}

%"Dissertation" for PhD, "Thesis" for master's
\documenttitle{Dissertation}

\degreename{Doctor of Philosophy}

% Use the wording given in the official list of degrees awarded by UCI:
% http://www.rgs.uci.edu/grad/academic/degrees_offered.htm
\degreefield{Integrated Composition, Improvisation and Technology}

% Your name as it appears on official UCI records.
\authorname{Isaac Otto Hayes}

% Use the full name of each committee member and full title 
% (e.g. Professor/Associate Professor).
\committeechair{Professor Mari Kimura}
\othercommitteemembers
{
  Professor Amy Bauer\\
  Professor Michael Dessen
}

\degreeyear{2023}

\copyrightdeclaration
{
  {\copyright} {\Degreeyear} \Authorname
}

% If you have previously published parts of your manuscript, you must list the
% copyright holders; see Section 3.2 of the UCI Thesis and Dissertation Manual.
% Otherwise, this section may be omitted.
% \prepublishedcopyrightdeclaration
% {
% 	Chapter 4 {\copyright} 2003 Springer-Verlag \\
% 	Portion of Chapter 5 {\copyright} 1999 John Wiley \& Sons, Inc. \\
% 	All other materials {\copyright} {\Degreeyear} \Authorname
% }

% The dedication page is optional
% (comment out to exclude).
\dedications
{
  To my mother and father.
}

\acknowledgments
{    
    I'd like to thank the members of my dissertation committee, Dr. Amy Bauer, Dr. Michael Dessen, and my committee chair Prof. Mari Kimura for their kindness and support and for being so forthcoming with their wisdom and experience.

    I would also like to thank advancement committee members Prof. Mark Dresser and Dr. Lisa Naugle, whose incisive commentary and tricky questions early on in the process helped contour my writing and creative work.

    I would like to thank The Band---Collin Felter, James Ilgenfritz, Steven Lewis, Jo\~{a}o Martins, Matthew Nelson, Bella Pepke, Atticus Reynolds, and Niloufar Shiri---as well as my technical supervisors Oliver Brown and Spencer Pepke. Without them I would be playing solo. 

    Certainly not least, I would like to thank Prof. Roscoe Mitchell, Prof. James Fei, and Prof. Fred Frith, who taught me how I might learn to be a better musician.
}


% Some custom commands for your list of publications and software.
\newcommand{\mypubentry}[3]{
  \begin{tabular*}{1\textwidth}{@{\extracolsep{\fill}}p{4.5in}r}
    \textbf{#1} & \textbf{#2} \\ 
    \multicolumn{2}{@{\extracolsep{\fill}}p{.95\textwidth}}{#3}\vspace{6pt} \\
  \end{tabular*}
}
\newcommand{\mysoftentry}[3]{
  \begin{tabular*}{1\textwidth}{@{\extracolsep{\fill}}lr}
    \textbf{#1} & \url{#2} \\
    \multicolumn{2}{@{\extracolsep{\fill}}p{.95\textwidth}}
    {\emph{#3}}\vspace{-6pt} \\
  \end{tabular*}
}

% Include, at minimum, a listing of your degrees and educational
% achievements with dates and the school where the degrees were
% earned. This should include the degree currently being
% attained. Other than that it's mostly up to you what to include here
% and how to format it, below is just an example.
%
% CV is required for PhD theses, but not Master's
% comment out to exclude
\curriculumvitae{

\textbf{EDUCATION}
  
  \begin{tabular*}{1\textwidth}{@{\extracolsep{\fill}}lr}
    \textbf{Doctor of Philosophy in} & \\
    \textbf{Integrated Composition, Improvisation and Technology} & \textbf{2023} \\
    \vspace{6pt}
    University of California, Irvine & \emph{Irvine, CA} \\
    \textbf{Master of Fine Arts in Music Performance and Literature} & \textbf{2018} \\
    \vspace{6pt}
    Mills College & \emph{Oakland, CA} \\
    \textbf{Bachelor of Arts in Psychology and Philosophy} & \textbf{2012} \\
    \vspace{6pt}
    University of Arkansas & \emph{Fayetteville, AR} \\
  \end{tabular*}

% \vspace{12pt}
% \textbf{RESEARCH EXPERIENCE}

%   \begin{tabular*}{1\textwidth}{@{\extracolsep{\fill}}lr}
%     \textbf{Graduate Research Assistant} & \textbf{2007--2012} \\
%     \vspace{6pt}
%     University of California, Irvine & \emph{Irvine, California} \\
%   \end{tabular*}

\vspace{12pt}
\textbf{TEACHING EXPERIENCE}

  \begin{tabular*}{1\textwidth}{@{\extracolsep{\fill}}lr}

    \textbf{Instructor of Record} & \textbf{2023} \\
    \vspace{6pt}
    \textbf{Teaching Assistant} & \textbf{2020--2023} \\
    \vspace{6pt}
    University of California, Irvine & \emph{Irvine, CA} \\
  \end{tabular*}

% \pagebreak

% \textbf{REFEREED JOURNAL PUBLICATIONS}

%   \mypubentry{Ground-breaking article}{2012}{Journal name}

% \vspace{12pt}
% \textbf{REFEREED CONFERENCE PUBLICATIONS}

%   \mypubentry{Awesome paper}{Jun 2011}{Conference name}
%   \mypubentry{Another awesome paper}{Aug 2012}{Conference name}

% \vspace{12pt}
% \textbf{SOFTWARE}

%   \mysoftentry{Magical tool}{http://your.url.here/}
%   {C++ algorithm that solves TSP in polynomial time.}

}

% The abstract was previously limited to a maximum of 350 words, 
% but the UCI manual at https://etd.lib.uci.edu/electronic/td2e#2.2.1.
% currently does not indicate that there is any word limit for the abstract
\thesisabstract
{
To date, there exists a startling lack of scholarly literature which attempts to systematically address novel notations for improvisers---particularly instances centering syntactically- and semantically- well-defined symbols. This lack of attention may be attributed, at least partially, to unclear definitions at the heart of the discourse and the lack of a rigorous typology of music notations generally. This dissertation, via a multi-pronged strategy, takes steps toward filling this lacuna. Chapter One provides a historical gloss which grounds twentieth- and twenty-first century performer/notation interaction in much earlier models. Here I demonstrate that overarching notions of notational ``fixity'' and ``openness,'' as well as notions of performance fidelity to a scored work are themselves fundamentally historically contingent. In the process, I highlight pivotal signposts in Western notation which mark important paradigm shifts in these conceptual categories. Chapter Two articulates a clear philosophical position with regard to notational semantic content, ``fixed'' and ``open'' notation, and the notion of the open score as first postulated in the 1950s and '60s. Here I challenge what I take to be conceits of the prevailing ``folk semiosis'' of music notation in order to begin developing a more analytically useful notation typology. To this end, I examine writings by Umberto Eco and Pierre Boulez, along with Gy\"{o}rgy Ligeti's „Neue Notation: Kommunikationsmittel oder Selbstzweck?”---by far the most lucid attempt to formulate such a typology. Chapter Three deploys concepts solidified in the previous chapter in service of a notation-centric analysis of two late-century work complexes: Anthony Braxton's \textit{Composition No. 76} and Horațiu Rădulescu's \textit{Das Andere}/Op. 89 (\textit{``Before the Universe was Born''}). Interrogating these works' related-but-disparate notation schemes grants new insights into notation's ability to mediate performer/composer agencies and to uniquely reflect composers' communities of practice and musico-philosophical commitments. Finally, Chapter Four thoroughly documents the author's efforts to develop and deploy a novel notation scheme for improvising musicians. This includes a discussion of several aspects of the design and preliminary implementation of \{O-G\} notation as well as of its use in a series of creative works intended to demonstrate its range and flexibility. This is followed by a frank assessment of the extent to which it was capable of fulfilling the author's initial desiderata and subsequent design criteria.
}


%%% Local Variables: ***
%%% mode: latex ***
%%% TeX-master: "thesis.tex" ***
%%% End: ***
