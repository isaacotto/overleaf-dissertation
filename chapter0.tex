% A SHORT INTRODUCTION
% Just for fun, maybe do one sentence per line.
% Just give an eloquent and broad explanation of the problem that I'm addressing and why I'm doing it, then give a somewhat detailed chapter summary. Follow the model of the abstract for like 1500 words and it will be fine.
% Is the goal analysis of these new musics? Or is the goal better understanding of NOTATION at large? Be extremely lucid here.

ELABORATE THE PROBLEM
Western music notation is functionally overloaded.
As the \textit{lingua franca} for so many disparate musical practices, it is pressed into service in a number of different capacities---many of them in conflict with each other.
For Western classical performers, it might serve as a recipe to be exactingly obeyed such that the resulting sounds might conform with appropriate precision to the composer's (real or imagined) wishes.
For the analyst, notation must be able to stand in robust, logically consistent relationships such that patterns may be observed, predictions made, and conclusions drawn.
For the transcriber, it must map one-to-one with observed sonic phenomena; able to represent minute sonic details well enough to be fit for archiving or later reproduction.
Finally, for the improviser, notation must serve to induce creative action, providing a material springboard for musical creativity while simultaneously delineating the boundaries of an authentic performance.

Given these requirements, Western notation has proved admirably metastatic, augmenting, paring down, or mutating its symbolic language depending on the needs of whichever community of musical practice adopted it.

    (When improvisation was in vogue, notation developed better ways of constraining era-appropriate improvisation. When, in the late 17th century, composers became more concerned with exactitude, they employed new methods of fixing the symbol-to-sound mapping.)
    -ex1
    -ex2


This was as true for new music in the 20th century as it was for that of the eighteenth. 
Owing to a number of disparate factors, the mid-to-late twentieth century saw a veritable explosion of novel notation techniques.
In a way these were no different from earlier adaptations---artists saw a new need and technology was developed to fill it.
Far and away most of the innovations around notation served to deliberately fold in notions of indeterminacy.
However, the true kernel of difference was that notation itself became a vector for artistry and expression to an extent theretofore unknown.
Composers, it seems, became increasingly interested in notation's power to affect not only a resultant sound-world, but new relationships between artist and interpreter.

[This is why mid-century neonotation practices of are particular interest.]

As a practicing composer-improviser myself, with a tortuous relationship to contemporary Western art music and jazz, I took particular interest in notation methods at the nexus of bleeding-edge mid-century-modernist composition and jazz/jazz-adjacent improvisation.
Specifically, those of the AACM (Anthony Braxton, Roscoe Mitchell, Wadada Leo Smith, George Lewis), those of the New York School (John Cage, Earle Brown, Christian Wolff, Morton Feldman), and those of the titans of the Euro-avant-garde (Karlheinz Stockhausen, Gyorgy Ligeti, Penderecki?) which...

My initial investigation into these methods, however, yielded deeply unsatisfying results.

Of course, there is no shortage of literature attempting to elucidate and contextualize these new practices. Scholars (since at least Umberto Eco) have recognized the significance of these fundamentally ``incomplete'' scores which require that players' creative contributions in order to be realized in performance. 
However, one finds that much of this literature lacks focus and precision. Many writers attempt to discuss novel notations (neonotation) without a robust notion of what differentiates the new from the old, and without a unifying narrative able to trace the ways openness has always been a part of notation's function and use.

[couple more sentences here]
[the multiplication of terminology: improvisation, indeterminacy, aleatory, openness]
[If I were being responsible, I'd have some concrete examples here---but those are also in the body of the text.]

This dissertation represents an attempt to fill this lacuna in scholarly literature pertaining to twentieth- and twenty-first-century notations oriented toward sonic indeterminacy and/or improvisation. In particular, my interest lies in notations which center syntactically- and semantically- well-defined symbols, as these often take a backseat to more ``radical'' methods which strip away notation's syntax and semantics to privilege the score-qua-art-object.

Though these represent a proportionally small niche when compared to more typical uses of Western notation, close study of these notation-forward musics has the potential to reveal hidden insight into the nature of our relationship to music notation at large.
[how so?]

DESCRIBE METHODOLOGY
My approach here will be multi-pronged. Though my eventual goal is the development of a more robust vocabulary with which to discuss contemporary neonotations, this task requires significant preparatory legwork.

CHAPTER ONE ...
Chapter one conducts a rapid historical survey in service of a narrative centering notational fixity and openness as core sites of innovation in Western art music notation. 
Beginning with Guido d'Arezzo and ending with the 1960s, I portray the history of Western notation as a gradual ebb and flow in the coupling between the printed page and its sonic products. 

CHAPTER TWO ...

CHAPTER THREE ...

CHAPTER FOUR ...

\begin{enumerate}
    \item NOTATION is confusing---it is pressed into service as a means of archiving sound, as a tool for analysis, as exacting recipe for performance, as inducement to improvise.   
\end{enumerate}
