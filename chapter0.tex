% A SHORT INTRODUCTION
% Just for fun, maybe do one sentence per line.
% Just give an eloquent and broad explanation of the problem that I'm addressing and why I'm doing it, then give a somewhat detailed chapter summary. Follow the model of the abstract for like 1500 words and it will be fine.
% Is the goal analysis of these new musics? Or is the goal better understanding of NOTATION at large? Be extremely lucid here.

%\begin{notestuff}
%Elaborate the problem.
%\end{notestuff}
\addcontentsline{toc}{chapter}{\protect\numberline{}Introduction}

Western music notation is overencumbered.
As the \textit{lingua franca} for so many disparate musical practices, it is pressed into service in a number of different capacities---many of them conflicting.
For Western classical performers, it must serve as a recipe to be more-or-less strictly obeyed such that the resulting sounds might conform with appropriate precision to the composer's (real or imagined) wishes.
For the analyst, notation must be able to stand in well-defined, logically consistent relationships such that patterns may be observed, predictions made, and conclusions drawn.
For the transcriber, it must map one-to-one with observed sonic phenomena; able to represent minute sonic details well enough to be fit for archiving or later reproduction.
Finally, for the improviser, notation must serve to induce creative action, providing a material springboard for musical creativity while simultaneously delineating the boundaries of an authentic performance.

Given these requirements, Western notation has proved admirably metastatic, augmenting, paring down, or mutating its symbolic language depending on the needs of whichever community of musical practice adopted it.
When medieval scholar-clergy needed a new way of archiving and transmitting sacred melodies, notation changed to serve as a pedagogical fixed point of reference.
When idiomatic improvised accompaniment was \textit{de rigueur}, composers and engravers brought in new symbols to clarify pieces' harmonic underpinnings. 
Likewise, when improvisation in Western art music fell out of vogue, notation reflected this by featuring new symbols to more precisely track composers' sound-concepts.

Notation's malleability held as much for new music in the twentieth century as it did for that of the eighteenth. 
Owing to a variety of factors, the mid-to-late twentieth century saw a veritable explosion of novel notation techniques; some quite durable but many which served only a single piece.
A significant proportion of these new notations (perhaps the majority) were developed in response to composers' growing desire for sonic indeterminacy---for scores which would generate predictably or unpredictably unique listening experiences each time they were realized. 
In a way this was the same as any earlier adaptation---artists saw a new need and developed technology to fill it.
There was a kernel of difference, however: during this period notation \textit{itself} became a vector for artistry and expression to an extent theretofore unknown.
Composers, it seems, became increasingly interested in notation's power to effect not only a resultant sound-world, but new relationships between artist and interpreter: thus they began to compose notation as well as music.
As a result, many of these new notations prove to be fertile ground for new scholarship; not only on their own merits but because they have the potential to provide insights into the practice of notating music more generally. 

As a practicing composer-improviser myself (and one with an atypically tortuous relationship to contemporary Western art music and jazz) I took particular interest in a subset of these methods, specifically those found near the swirling nexus between mid-century avant-garde ``classical'' music and jazz (or jazz-adjacent) improvisation---especially those of the AACM (Anthony Braxton, Roscoe Mitchell, Wadada Leo Smith, George Lewis).
Studying these, in turn, led to greater awareness of their European and American ``concert music'' peers and forebears: artists in the New York School (John Cage, Earle Brown, Christian Wolff, Morton Feldman) and the titans of Euro-modernism (Karlheinz Stockhausen, Gy\"orgy Ligeti, Helmut Lachenmann, Horațiu R\u{a}dulescu, to name a few.)
As the purview of my study expanded, I became interested not only in the unique forms of their symbols themselves, but also in their motivations for encoding their desired sounds and processes in new ways.
Some important questions arose: What factors must be considered when designing a new notation schemes? 
How does one construct a notation so as to balance predictable form with an ever-changing sonic surface?
What differentiates notations which properly \textit{encode} from those which seek only to effect ekphrasis, a necessarily imperfect translation from one aesthetic domain to another? 
My initial investigation into these questions, however, yielded deeply unsatisfying results. 

%\begin{notestuff}
%    Problematize the literature.
%\end{notestuff}
Of course, there is no shortage of literature attempting to elucidate and contextualize these new practices. 
Scholars (since at least Umberto Eco) have recognized the novelty and significance of these fundamentally ``incomplete'' scores which require players' creative contributions in order to be realized in performance. 
However, one finds that much of this literature lacks focus and precision. 
Far more attention seems to be paid to new notations' aesthesis and physical trace than is paid to their function.
Many writers attempt to discuss novel notations without a robust notion of what differentiates the new from the old, and without a unifying narrative able to account for the ways openness has always been a part of notation's function and use.
Further, writers routinely conflate composers' underlying philosophical commitments with the tools they use to inscribe their music.
Thus, categories of notation are unnecessarily multiplied (``aleatoric notation,'' ``indeterminate notation,'' ``improvisatory notation'' and worst of all ``graphic notation'') yet lack unambiguous definitions that would distinguish them from earlier forms of notation and from each other.
Finally, there remains considerable need for comparative studies which would interrogate two or more novel approaches to notation in tandem so as to reveal their functional distinctions.
Though composers who favor neonotations develop their tools with particular, personal aesthetic goals in mind, many exhibit significant overlap in their means and ends (in addition to their noteworthy differences) which all too often go unconsidered.
This, however, would necessitate a significant expansion and/or refinement of the language used to describe notation's form and function.

%In other words, discourse pertaining to neonotation in particular (and perhaps notation generally) suffers from a marked lack of standard practices.
    
This dissertation is thus an attempt to address, in part, these lacunae in the study of ``open notations,'' i.e., of notations oriented toward finer mediation of sonic indeterminacy.
To be more specific, my interest principally lies in notations which center syntactically and semantically well-defined symbols, as these seem to frequently take a backseat to more ``radical'' methods which deliberately strip away notation's syntax and semantics.\footnote{...though to be clear these ``asemantic'' scores, too, deserve greater scrutiny.}
Though these well-structured ``open'' scores are relatively niche compared to those which employ more typical forms of Western notation, by way of careful study they stand to reveal new insights into our relationships to music notation at large.
Though my eventual goal is the development of a more robust vocabulary with which to discuss contemporary neonotations, this task requires significant preparatory legwork and a multilateral approach.

%\begin{notestuff}
%    Chapter breakdown.
%\end{notestuff}
%\paragraph{Chapter One} % Remove these \paragraphs before go time.
Chapter One conducts a rapid historical survey, developing a narrative which locates notational fixity and openness as core sites of innovation in Western art music notation. 
Beginning with Guido d'Arezzo and ending in the 1960s, I portray the history of Western notation as a gradual ebb and flow in the degree of coupling between the printed page and its sonic products. 
Specifically, I note the way that the form and function of notation have always responded to the particular needs of its user-base, though, crucially, that form and function are not always neatly tied together.
Where significant advances in either occur, I examine specific instances where this change is made manifest, which range from the development of \textit{cantare super librum} (which I take to be one of the earliest modern literate/improvisatory practices), to the decline of notation's ``openness'' in the late-eighteenth and nineteenth centuries, to the Afro-diasporic return to notation for improvisers in the mid-twentieth.
In addition, I briefly examine some of the first (predominantly New York School) forays into the new notations at the heart of this assessment. 
Here I render notational openness and fixity as first emergent and later deliberately exploited musical parameters, always already linked to the very notion of notated music.

%\paragraph{Chapter Two}
Chapter Two uses concepts explored in the first chapter to examine several aspects of notation's function in detail, ultimately taking steps toward a more robust typology of (open) notations
Beginning from first principles, I pose a new model by which we might conceive of notation's semantic content---i.e., what these symbols represent and how they convey their message.
So as to facilitate the development of a more useful critical lexicon, I challenge what I take to be the prevailing (often insufficiently-articulated) ``folk-semiosis'' pertaining to the function of music notation generally.
Following this, I examine essays by three authors, Umberto Eco, Pierre Boulez, and Gy\"{o}rgy Ligeti (all roughly contemporary with the mid-century notation boom), who each seek to address the roles and goals of various open notations. 
In particular, I spend significant time on Ligeti's 1965 „Neue Notation: Kommunikationsmittel oder Selbstzweck?,” a woefully under-appreciated essay in which Ligeti, a scholar-composer situated at the bleeding edge of Western concert music innovation, lends unique perspective to questions of notation's function.
Most importantly, Ligeti examines a number of contemporary neonotational works and, in so doing, puts forward a new typology so as to accommodate these new pieces.
Finally, based on these insights I propose two new descriptors, ``traversal'' and ``hybridity,'' used to describe the ways composers combine and move through notations at varying degrees of fixity and with varying levels of semanticity. 

%\paragraph{Chapter Three}
Chapter Three demonstrates an application of this new descriptive paradigm, examining two ``work complexes'' by late-century composers Anthony Braxton and Horațiu R\u{a}dulescu through the lens of a modified Ligetian typology.
Braxton and R\u{a}dulescu, two innovators belonging to very different communities of musical practice, each employ complex, bespoke notation schemes which, to speak broadly, deliberately parametrize notational fixity and openness in various ways.
Here, I attempt to lead by example, conducting a notation-centric analysis comparing the two composers' systems on a symbol-for-symbol basis and describing what I take to be the most salient points of overlap and departure between them.
Through this analysis, I draw conclusions pertaining not only to Braxton's and R\u{a}dulescu's writing methods, but also to their underlying philosophical commitments and to the unexamined potential of well-defined ``open'' notations generally.

%\paragraph{Chapter Four}

Finally, Chapter Four serves to document the author's yearslong creative investigation of the topics explored in previous chapters; specifically, the design and implementation of \{O-G\} (``Otto-Glyphs'') a novel notation scheme for improvising musicians.
I begin by discussing the system's initial catalysts: dissatisfaction with some of my formative experiences with scored improvisation and my subsequent exposure to a number of artists who had succeeded in creatively circumventing these same problems.
I continue by summarizing \{O-G\}'s nascence, addressing a number of important milestones including the first piece to be formally composed in the scheme as well as the formal ``instruction manual'' which eventually became the core tool by which I inducted new players into the system.
The system (and my pedagogical method) would ultimately be put to the test in \textit{I Die Each Time I Hear the Sound}, a concert of original works composed either entirely in \{O-G\} or with well-integrated \{O-G\}/traditional notation.
I thus conclude by summarizing several of the more noteworthy compositions from this series before conducting a broad-level assessment of \{O-G\}'s successes and failures according to the criteria delineated at the project's outset.



