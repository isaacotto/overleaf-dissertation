% A SHORT INTRODUCTION
Just give an eloquent and broad explanation of the problem that I'm addressing and why I'm doing it, then give a somewhat detailed chapter summary. Follow the model of the abstract for like 1500 words and it will be fine.

Western music notation is functionally overloaded. As the \textit{lingua franca} for so many disparate musical practices, it is pressed into service in a number of different capacities---many of them in conflict with each other. For Western classical performers, it might serve as a recipe to be exactingly obeyed such that the resulting sounds might conform with some precision to the composer's (real or imagined) wishes. For the analyst, notation must be able to stand in logically consistent relationships such that patterns may be observed, predictions made, and conclusions drawn. For the transcriber, it must map one-to-one with observed sonic phenomena; able to represent minute sonic details well enough to be fit for archiving or later reproduction. Finally, for the improviser, notation must serve to induce creative action, providing a material springboard for musical creativity while simultaneously delineating the boundaries of an authentic performance.

Of course, Western notation has proved quite flexible; able to adapt... it has always adapted. When improvisation was in vogue, notation developed better ways of constraining era-appropriate improvisation. When, in the late 17th century, composers became more concerned with exactitude, they employed new methods of fixing the symbol-to-sound mapping.

This is as true in the 20th century as it was in the eighteenth. 
The 50s and 60s saw a veritable explosion of novel notation techniques.
In a way this was no different from earlier adaptations---there was a new need and it was filled.
Far and away most of the innovations around notation served to deliberately fold in notions of indeterminacy.
The real difference was that notation itself became a vector for artistry and expression to an extent theretofore unknown.

This did not go unnoticed by scholars who quickly saw an avenue for speculation and analysis. However, much of this literature lacks focus and precision. It attempts to discuss novel notations (neonotation) without a robust notion of what differentiates the new from the old, and without a unifying narrative tracing the ways openness has always been a part of notation's function and use.

This dissertation represents an attempt to fill a lacuna in scholarly literature pertaining to novel notations for improvisers---particularly instances centering syntactically- and semantically- well-defined symbols.

CHAPTER ONE ...

CHAPTER TWO ...

CHAPTER THREE ...

CHAPTER FOUR ...

\begin{enumerate}
    \item NOTATION is confusing---it is pressed into service as a means of archiving sound, as a tool for analysis, as exacting recipe for performance, as inducement to improvise.   
\end{enumerate}